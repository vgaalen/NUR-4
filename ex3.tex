\section{Spiral and elliptical galaxies}

%\textcolor[red]{insert code}

\subsection*{a}
For this exercise I classified galaxies into two types: spirals and ellipticals, using logistic regression.

First the data has to be prepared; logistic regression works best on a dataset with mean 0 and standard deviation 1.
Therefore we rescale the given dataset, resulting in the following:
\lstinputlisting[lastline=10, caption=The first 10 entries in the dataset after rescaling.]{3a.txt}

The full distributions than like this (figure \ref{fig:rescaled_data}):
\begin{figure}
    \centering
    \includegraphics[width=0.8\textwidth]{fig3a.png}
    \caption{The data after rescaling.}
    \label{fig:rescaled_data}
\end{figure}

\lstinputlisting[lastline=42, caption=Code used for Exercise 3a]{ex3.py}



\subsection*{b}
Now we perform the actual regression by minimizing a sigmoid cost function using the Downhill Simplex algorithm.
In order to find the parameters that best describe a galaxy's type, only two parameters are used at a time.
All possible combinations are then compared in figure \ref{fig:logistic_regression}.

\begin{figure}
    \centering
    \includegraphics[width=0.8\textwidth]{fig3b.png}
    \caption{The results of the logistic regression.}
    \label{fig:logistic_regression}
\end{figure}

This plot shows that $\kappa_{\text{CO}}$ and the color best describe the type of galaxy as this combination achieves the lowest cost.

\lstinputlisting[firstline=44,lastline=200, caption=Code used for Exercise 3b]{ex3.py}


\subsection*{c}
In order to see how well the logistic regression works, I computed the true positives, true negatives, false positives, false negatives, and F1-Score for each set of parameters:
\lstinputlisting{3c.txt}

This shows again that the first set of parameters is best suited for this application.

\lstinputlisting[firstline=202, caption=Code used for Exercise 3c]{ex3.py}

