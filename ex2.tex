\section{Calculating forces with the FFT}

\lstinputlisting[lastline=35, caption=Code Shared Among the Subquestions of Exercise 2]{ex2.py}
%\textcolor[red]{insert code}

\subsection*{a}
Tracking the forces between a large number of particles can be computationally expensive.
One way to get around this is to use the Cloud-In-Cell method.
In this method all particles are assigned to a grid cell and the forces on a particle are calculated for this cell as a whole.
Beyond the lower number of calculations required, this method also has the advantage of giving a field instead of discrete parameters.
This allows the use of the Fast Fourier Transform (FFT) to speed up the calculation of the forces.

For the gravitational force the density contrast is the important field (and not the density itself).
This is defined as $\delta(\mathbf{x}) = \frac{\rho(\mathbf{x})}{\bar{\rho}} - 1$.
Figure \ref{fig:density_contrast} shows the density contrast for four slices in the 16x16x16 grid used here.
\begin{figure}
    \centering
    \includegraphics[width=0.8\textwidth]{fig2a.png}
    \caption{Density contrast for four slices in the 16x16x16 grid. All parameters are in arbitrary units}
    \label{fig:density_contrast}
\end{figure}

\lstinputlisting[firstline=37,lastline=58, caption=Code used for Exercise 2a]{ex2.py}



\subsection*{b}
To convert the density contrast into gravitational forces, the Poisson equation is used:
\begin{align}
    \nabla^2 \phi = \nabla \cdot \nabla \phi = 4 \pi G \bar{\rho} (1 + \delta).
\end{align}
As we only need the spatial dependance in this field, we can simplify this to $\nabla^2 \phi \propto \delta$.
This can be easily solved in Fourier space:
\begin{align}
    \nabla^2 \Phi \propto \delta \xrightarrow{\text{FFT}} k^2 \tilde{\Phi} \propto \tilde{\delta} \xrightarrow{rewrite} \tilde{\Phi} \propto \frac{\tilde{\delta}}{k^2}
    \xrightarrow{iFFT} \Phi.    
\end{align}

An FFT implementation using the Cooley-Tukey algorithm then gives the gravitational potential $\Phi$, as can be seen in figure \ref{fig:gravitational_potential}.
The intermediate result of the fourier transformed density contrast ($\tilde{\Phi}$) is shown in figure \ref{fig:fourier_potential}.

\begin{figure}
    \centering
    \includegraphics[width=0.8\textwidth]{fig2b_fourier.png}
    \caption{The fourier transfer of the density contrast $~\Phi$ for four slices in the 16x16x16 grid.}
    \label{fig:fourier_potential}
\end{figure}

\begin{figure}
    \centering
    \includegraphics[width=0.8\textwidth]{fig2b_grav.png}
    \caption{Gravitational potential $\Phi$ for four slices in the 16x16x16 grid.}
    \label{fig:gravitational_potential}
\end{figure}

\lstinputlisting[firstline=60, caption=Code used for Exercise 2b]{ex2.py}




