\section{Simulating the solar system}

\textcolor[red]{insert code}

\subsection*{a}
In this exercise I simulate the orbits of the eight planets in the solar system.
The easiest set of initial parameters is one sampled from reality.
In order to get the initial positions and velocities of the planets, I used the \texttt{astropy} package.
Its \texttt{get\_body\_barycentric()} function gets this information from JPL's DE430 ephemerides \textcolor[red]{reference}.

These positions are shown in figure \ref{fig:initial_positions}.

\begin{figure}
    \centering
    \includegraphics[width=0.8\textwidth]{fig1a.png}
    \caption{Initial positions of the planets in the solar system.}
    \label{fig:initial_positions}
\end{figure}



\subsection*{b}
In order to simulate the orbits, I have to integrate equations of motion incorporating gravitational forces.
Here a heavily simplified approach is used.
Only the Newtonian gravitational force between the sun and planets is incorporated (so planet-planet attraction is neglected).

For orbits in particular, it is important to have an integrator that conserves energy as otherwise the orbits will spiral outwards or inwards.
The leapfrog integrator is a good choice for this, as it is simple and conserves energy.
This integrator 'leaps' the position and velocities separately with a half time step between them:
\begin{align}
    \mathbf{v}_{1/2} &= \mathbf{v}_0 + \frac{1}{2} \mathbf{a}_0 \Delta t \\
    \mathbf{x}_{i+1} &= \mathbf{x}_i + \mathbf{v}_{i+1/2} \Delta t \\
    \mathbf{v}_{i+3/2} &= \mathbf{v}_{i+1/2} + \mathbf{a}_{i+1} \Delta t
\end{align}

Integrating over 200 years with timesteps of 0.5 days, gives the orbits shown in figure \ref{fig:orbits}.
\begin{figure}
    \centering
    \includegraphics[width=0.8\textwidth]{fig1b.png}
    \caption{Orbits of the planets in the solar system. Integrated over 200 years with a timestep of 0.5 days using the leapfrog integrator.}
    \label{fig:orbits}
\end{figure}
\begin{figure}
    \centering
    \includegraphics[width=0.8\textwidth]{fig1b-1.png}
    \caption{The same as figure \ref{fig:orbits}, but zoomed in on the inner planets.}
    \label{fig:orbits-zoom}
\end{figure}



\subsection*{c}
\todo{1c}